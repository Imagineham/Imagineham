%---------change this every homework
% place your email id between the braces so that your homework has a name
\def\yourname{Hamlet}
% -----------------------------------------------------
\def\homework{2} % 0 for solution, 1 for problem-set only
\def\duedate{Tuesday, January 26 at 12 noon}
\def\duelocation{via \href{https://www.gradescope.com/courses/80616/}{gradescope}}
\def\hnumber{2}
\def\prof{Andrew Drucker, Lorenzo Orecchia}
\def\course{\href{https://canvas.uchicago.edu/courses/25977}{CMSC 27200 - Winter 2021}}
%-------------------------------------

\documentclass[10pt]{article}
\usepackage[colorlinks,urlcolor=blue]{hyperref}
\usepackage[osf]{mathpazo}
\usepackage{amsmath,amsfonts,graphicx}
\usepackage{latexsym}
\usepackage{subfig}
\usepackage{algpseudocode}
%\usepackage{enumitem}
\usepackage{algorithm}
\usepackage{listings}
%\usepackage[top=1in,bottom=1.4in,left=1.5in,right=1.5in,centering]{geometry}
\usepackage{fullpage}
\usepackage{color}
\definecolor{mdb}{rgb}{0.3,0.02,0.02} 
\definecolor{cit}{rgb}{0.05,0.2,0.45} 
%\pagestyle{myheadings}
\markboth{\yourname}{\yourname}

\thispagestyle{empty}

\newenvironment{proof}{\par\noindent{\it Proof.}\hspace*{1em}}{$\Box$\bigskip}
\newcommand{\qed}{$\Box$}
\newcommand{\alg}[1]{\mathsf{#1}}
\newcommand{\handout}{
   \renewcommand{\thepage}{H\hnumber-\arabic{page}}
   \noindent
   \begin{center}
      \vbox{
    \hbox to \columnwidth {\sc{\course} --- \prof \hfill}
    \vspace{-2mm}
    \hbox to \columnwidth {\sc due \MakeLowercase{\duedate} \duelocation\hfill {\Huge\color{mdb}H\hnumber.\yourname}}
      }
   \end{center}
   \vspace*{2mm}
}
\newcommand{\solution}[1]{\medskip\noindent{\color{cit}\textbf{Solution:} #1}}

\newcommand{\bit}[1]{\{0,1\}^{ #1 }}
%\dontprintsemicolon
%\linesnumbered
\newtheorem{problem}{\sc\color{cit}problem}
\newtheorem{lemma}{Lemma}
\newtheorem{theorem}{Theorem}
\newtheorem{definition}{Definition}
\newtheorem{claim}{Claim}


\begin{document}
\handout
\begin{itemize}
\item The assignment is due at Gradescope on \duedate. 

\item You can either type your homework using LaTex or scan your handwritten work. We will provide a LaTex template for each homework. If you writing by hand, please fill in the solutions in this template, inserting additional sheets as necessary. This will facilitate the grading.

\item You are permitted to study and discuss the problems with 2 other students (per problem; any section).  However, {\em you must write up your own solutions, in your own words}. Do not submit anything you cannot explain. If you do collaborate with any of the other students on any problem, please do list all your collaborators in your submission for each problem. 

\item Similarly, please list any other source you have used for each problem, including other textbooks or websites.


\item {\em Show your work.} Answers without justification will be given little credit.
\end{itemize}

%%%%%%%%%%%%%%%%%%%%%%%%%%%%%%%%%%%%%%%%%%%

%%%%%%%%%%%%%%%%%%%%%%%%%%%%%%%%%%%%%%%%%%%
\newpage
\begin{problem}[25 points]
Solve exercise 3 in Chapter 4 in the Kleinberg-Tardos textbook.  (trucking)
\end{problem}

\solution{
  Collaborated with: Yael Sulkin, Bryan Lee \\

\paragraph{Algorithm} The greedy algorithm this proof will be proving is the following:
      
\begin{algorithmic}
      \State Let all trucks have maximum weight, $W$ 
      \State Let the number of trucks sent to Boston, $numTrucks$. 
      \State Let the truckweight at $k$th iteration be $truckWeight$. 
      \State Let $i$ be the package that most recently arrived, woth weight of i be $w_i$ \\

      \While {packages arrive}  
        \If {$truckWeight + w_i < W$} 
          \State update $truckweight = truckweight + w_i$
        \Else 
          \State increase $numTrucks$ by 1
          \State reset $truckWeight$ to $0$
        \EndIf
      \EndWhile
\end{algorithmic}

\paragraph{Proof} We will prove this algorithm inductively 
}


\newpage
\medskip\noindent{\color{cit} Extra Space for your solution}

%%%%%%%%%%%%%%%%%%%%%%%%%%%%%%%%%%%%%%%%%%%

%%%%%%%%%%%%%%%%%%%%%%%%%%%%%%%%%%%%%%%%%%%
\newpage
\begin{problem}[25 points]
Solve exercise 5 in Chapter 4 in the Kleinberg-Tardos textbook.  (cell phone towers)

Please note!  Here and elsewhere, when the authors (or your instructors) say ``give an algorithm'' without further instructions, they mean 
``give an algorithm AND prove that it is correct and runs in polynomial time''.  This should be assumed henceforth.
\end{problem}
\solution{
  Collaborated with: Yael Sulkin, Bryan Lee\\

  \begin{algorithmic}
    \State Let $h$ be in set $H$ of all houses along the road.
    \State Let $x_h$ be the locatin of house $h$ where $x_h$ = 0 if at westernmost point and the easternmostpoint is at $x_h$ max. 
    \State Let $B$ represent the set of base towers placed

    \While {$h \in H$}
      \If {$x_h + 4 \geq$ easternmost point}
        \State place base on easternmost point
        \State add base to set B
        \State delete all houses in $H$ s.t. $x_4 + 4 \geq$ easternmost point
      \Else 
        \State place a base in $x_h + 4$
        \State add base to set B
        \State delete all houses in H s.t. $x_h' \leq x_h + 8 miles$
      \EndIf
    \EndWhile

  \end{algorithmic} 
 
  \paragraph{}We know our algorithm runs on polynomial time because our loop iterates over the set of houses once, returning the set of bases after reaching the last house. 
  Our algorithm runs on $O(n)$ time by definition. 

  \paragraph{Proof} Let set $A$ be the set of bases placed by our algorithm and set $O$ be the set of bases placed 
  by the optimal algorithm. If our set $A$ is not optimal and we are looking for the minimal amount of bases then 
  we know that if $|A| = n$ and $|O| = m$ then $m < n$. If $m < n$ then we know there exists some base, $b$ in $A$ such that
  it either covers $0$ houses or there also exists some $b'$ such that $b'$ can be moved to cover the houses within $b$. We will continue this 
  proof using case analysis and proof by contradiction. 

  \paragraph{} In the case where $b$ in $A$ covers $0$ houses, this, in fact, contradicts the algorithm for $A$ which states 
  "while there exists house $h \in H$". This case is impossible for our proposed greedy algorithm. 

  \paragraph{} In the case where $b'$ exists in $A$ such that $b'$ can be moved, let $b$ be located at distance $x$ and $b'$ be located at distance $x'$ 
  such that $x \neq x'$ and $x, x' \geq 0$. According to our case statement, there exists $h$ in ${x - 4, x + 4}$ such that $b'$ can be moved to cover 
  ${x - 4, x + 4} \cup {x' - 4, x' + 4}$ or ${x - 4, x + 4, x' - 4, x' + 4}$. $b'$ will have to at most cover 
  $|(x - 4) - (x' + 4)| = |x - x' - 8|$.
  If the distance $b'$ has to cover is greater than 8 this contradicts our givens. If $|x - x' - 8| = 0$ this means base $b$ and base $b'$ are exactly 8 meters apart. Let $b$ be west of $b'$. 
  This implies there exists some $h$ such that $x_h$ is exactly halfway between $b$ and $b'$. However, this is impossible because according to our algorithm $b'$ was placed 4 miles east of house $h \in H$ yet
  $h$ was not in $H$ at that iteration because $x_h$ was covered by base $b$ and therefore deleted from available input. 
}


\newpage
\medskip\noindent{\color{cit} Extra Space for your solution}

%%%%%%%%%%%%%%%%%%%%%%%%%%%%%%%%%%%%%%%%%%%

%%%%%%%%%%%%%%%%%%%%%%%%%%%%%%%%%%%%%%%%%%%
\newpage
\begin{problem}[25 points]
The Running Sums problem is defined as follows:

\vspace{1 em}


\textbf{Input:} a sequence $(a_1, \ldots, a_n)$, where each $a_i$ is either 1 or -1.  

\vspace{1 em}

\textbf{Desired output:} a sequence $(b_1, \ldots, b_n)$, where each $b_i$ is either 0 or 1.  

Your goal is to \textbf{minimize} $\sum_{1 \leq i \leq n} b_i$, subject to the \textbf{constraint} that

\[    \sum_{1 \leq i \leq j} (a_i + b_i) \ \geq \ 0 \ ,   \ \ \ \ \ \ \ \ \   \mathrm{for \ all \ \ \ }j \in \{1, 2, \ldots, n\} \ .  \]

Give an algorithm for this problem that, for full credit, should run in $O(n)$ steps.



\end{problem}
\solution{
Collaborated with: Yael Sulkin and Bryan Lee

\begin{algorithmic}
  \State Let $b$ start off as $0$
  \State Let $J$ represent the running sum of $a_i$ and $b_i$
  \State Let $B$ represent the running sum of $b_i$ 

  \While {$a_i$ in $a_1 \ldots a_n$}
    \If {$J > 0$}
      \State $b_i$ remains $0$
      \State Update $J$ and add $a_i$ and $b_i$ 
      \State Keep $B$ the same
    \Else 
      \State $b_i$ can be derived by the equation $b_i = \frac{a_i}{2} + \frac{1}{2}$
      \State Update $J$ by add $a_i$ and $b_i$
      \State Update $B$ by adding $b_i$
    \EndIf
  \EndWhile

\end{algorithmic}

\paragraph{} This algorithm runs 
\paragraph{Proof} 

}

\newpage
\medskip\noindent{\color{cit} Extra Space for your solution}

%%%%%%%%%%%%%%%%%%%%%%%%%%%%%%%%%%%%%%%%%%%

%%%%%%%%%%%%%%%%%%%%%%%%%%%%%%%%%%%%%%%%%%%
\newpage
\begin{problem}[25 points]
In the Hopping Game, there is a sequence of $n$ spaces.  You begin at space 0 and at each step, you can hop 1,2,3, or 4 spaces forward. However, some of the spaces have obstacles and if you land on an obstacle, you lose. 

Give a greedy algorithm which, given an array $A[1, \ldots, n-1]$ of Boolean values with $A[i]$ indicating the presence/absence of obstacle at position $i \in [1, n-1]$, find the minimum number of hops needed to reach space $n$ without losing, if it is possible to do so.  (We assume spaces 0 and $n$ are obstacle-free, and are not part of the input.)  Prove that your algorithm is correct. For full credit, your algorithm should run in time $O(n)$.
\end{problem}

\solution{ 
  Collaborated with: Yael Sulkin

\begin{algorithmic}
  \While {for every input $a_i$ in $A$}
    \If {$a_{i + 4}$ is $0$}
      \State add to hops counter
      \State change location to $a_{i + 4}$
    \Else 
      \If {$a_{i + 3}$ is $0$}
        \State add to hops counter
        \State change location to $a_{i + 3}$
      \Else 
        \If {$a_{i + 2}$ is $0$}
          \State add to hops counter
          \State change location to $a_{i + 2}$
        \Else 
          \If {$a_{i + 1}$ is $0$}
            \State add to hops counter
            \State change location to $a_{i + 1}$
          \Else 
            \State Return hops counter, Terminate loop. Reaching N is not possible
          \EndIf
        \EndIf
      \EndIf
    \EndIf
  \EndWhile 


\end{algorithmic}

\paragraph{} This algorithm has runtime O(n) since we work our way through array $A$ from left to right once
and never loop over any square multiple times. Our algorithm's runtime is proportional to the size of $A$ which by definition is 
O(n). 

\paragraph{Proof}
}


\newpage
\medskip\noindent{\color{cit} Extra Space for your solution}



\end{document}
